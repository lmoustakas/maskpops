\message{ !name(clashresolved.tex)}\documentclass{emulateapj}
\usepackage{natbib}
%\usepackage{color}
%\usepackage{xspace}
\citestyle{aa}

\shorttitle{Galaxy Formation Models \& Spatially Resolved Stellar Populations} 
\shortauthors{Moustakas et al.}

\begin{document}

\message{ !name(clashresolved.tex) !offset(-3) }

\bibliographystyle{apj}

\submitted{Submitted to Astrophysical Journal}

%\def\ltsima{$\; \buildrel < \over \sim \;$}
%\def\lsim{\lower.5ex\hbox{\ltsima}}
%\def\gtsima{$\; \buildrel > \over \sim \;$}
%\def\gsim{\lower.5ex\hbox{\gtsima}}

\title{Mapping Galaxy Formation Stories in Distant Spatially Resolved Strongly Lensed Galaxies} 

\author{Leonidas A. Moustakas\altaffiltext{1}}
\author{Genevieve Graves\altaffiltext{2}}
\author{John Moustakas\altaffiltext{3}}
\altaffiltext{1}{JPL}
\altaffiltext{2}{UC Berkeley}
\altaffiltext{3}{UCSD}

\begin{abstract}
An abstract here. 
\end{abstract}
\keywords{gravitational lensing}

\section{Introduction} 

% (\citealt{bunker:00})


Gradients in Spiral Galaxies

Metallicity gradients: I think it is pretty well established that
spiral galaxies have substantial metallicity gradients, with
metal-rich centers and metal-poor outer parts.  People have done this
looking at stellar metallicity gradients with colors in other galaxies
\citep[e.g.][]{Bell:00}, with stellar spectra in our own galaxy
\citep[e.g.][]{Ivezic:08}, and by mapping the metallicity gradients
of HII regions in star-forming galaxies \citep[e.g.][]{Zaritsky:94}. 
  
Age gradients: There are multiple studies that suggest that spirals
also have significant age gradients, with older stars in the center
and younger stars in the outer parts.  e.g., Bell \& de Jong (2000)
fit SEDs to radially resolved optical and near-IR photometry and find
the centers of galaxies to be old and metal-rich, while the outer
parts are younger and more metal-poor.  Kauffmann et al. (2006) find
that many massive bulge-dominated galaxies have blue outer regions,
which appear to show residual star formation even after star formation
has ceased in the inner part of the galaxy.

Gradients in Early Type Galaxies


\citep{Somerville:2003aa}\citep{Ferguson:2004qw}

\citep{Moustakas:1998tn}

Metallicity gradients: It is well-established that early type galaxies
show strong metallicity gradients, with metal-rich centers and
metal-poor outer parts.  The SAURON team (Scott, Cappellari et
al. 2009) recently published this in an interesting and somewhat novel
way---plotting [Z/H] as a function of the local escape velocity in the
galaxy (essentially a radial gradient, with high Vesc in the center).
There are many other studies that show similar things, both with
spectral absorption line strengths and colors.

Age gradients: These are more debated.  The Scott et al. (2009) paper
mentioned above find age gradients that are flat in some galaxies,
while other show youngish central ages and older ages farther out.
Another relevant paper is Sanchez-Blazquez et al. (2007), who find
weak or no age gradient in 10/11 early types, while one have a very
young center.  Interestingly, the weak positive age gradients,
combined with the negative metallicity gradients, means that the
age-metallicity degeneracy acts to keep galaxy color (and
line-strength) gradients weaker than they otherwise might be.

Abundance ratio gradients: There do not seem to be strong [alpha/Fe]
gradients in early types.  In the papers quoted above, Scott et
al. (2009) find no [alpha/Fe] gradients and Sanchez-Blazquez et
al. (2007) find no gradients, weak positive gradients, or weak
negative gradients in their galaxies.

Gradients beyond the half-light radius: In general, these studies do
not go beyond the half-light radius of the galaxy.  Some work that I
am doing right now tracing stellar population gradients out to 2-3
Reff essentially agrees with these studies: Age gradients vary from
galaxy-to-galaxy, are usually weak, but with some galaxies showing
young cores.  All galaxies show strong negative gradients in total
metallicity.  [alpha/Fe] varies little with radius.  Interestingly,
carbon-enhancement ([C/Fe]) seems to scale strongly with
radius---galaxies have much higher [C/Fe] in their centers that in
their outskirts.

At high redshift -- stuff happens. 

It is of course much harder to study population gradients at high
redshift.  Rigby et al. (2011) and Wuyts et al. (2010) have SEDs and
spectroscopy for a lensed galaxy at z=1.7, but don't try to extract
spatially-resolved information.  There was a Rigby HST proposal in
Cycle 18 to do resolved SED fitting for a lensed arc (might well be
the same one) which I think got time, but I don't see the paper out
yet.

Theory

Spiral galaxies: I don't know much about formation models for spiral
galaxies.  Bell \& de Jong (2000) propose a scenario in which star
formation happens early and rapidly in high-density regions of the
galaxy (i.e., the center) and proceeds more slowly out at large radius
where the gas densities are low, hence leaving low-lever star
formation in the outer parts of the galaxy at late times.  There is a
huge literature on the Milky Way star formation history and detailed
chemical evolution models with which I am not very familiar.  Some
names that spring to mind are Brad Gibson, Rok Roskar, Daisuke Kawata,
Chiaki Kobayashi.  I'm sure I'm missing some big ones.

Early-type galaxies: There are limited numbers of chemical evolution
models for early type galaxies (Gibson et al. 2007, Pipino et
al. 2010), and they tend to be trying to reproduce existing
observations, not making strong new predictions.  An interesting
recent paper is Oser et al. (2010).  They model massive early type
galaxies using a hydro-code super-posed on a zoom-in of a cosmological
N-body simulation.  Their galaxies wind up containing two
sub-populations of stars, those that formed "in situ" in the massive
halo, and those that formed in satellites outside Rvir of the main
halo, which were then accreted at later times.  The in situ stars
dominate the central part of the galaxy, while the accreted stars wind
up at much higher radius, thus setting up a strong gradient in the
origin of the stars.  Interestingly, both the in situ stars and the
accreted stars tend to be old.  There is often a little dribble of
residual in situ star formation in their models of less massive
galaxies, which would tend to make the central population younger.
Also, you'd expect strong differences in the metallicities of the two
components, with the in situ stars showing much higher metallicities
than those that were accreted from small satellites, due to the
mass-metallicity relation and the greatly reduced ability of small
galaxies to hold onto metal-enriched SN ejecta.


The power of strong gravitational lensing as a tool towards a broad
range of astrophysical discoveries. (Reference to some lensing school
notes, eg the Saas Fee lectures). The lensing object (REF), magnified
background sources (\citealt{bunker:00}), and indeed for cosmology
(Coe \& Moustakas, submitted). Lots of references here for background,
nothing too profound.

To these ends, since the first strong lens was discovered by
\citet{walsh:79}, both serendipity and systematic programs have
uncovered several hundred strong lenses by the present day. Broadly
speaking, these have been found through one of two techniques, whether
serendipitous or through a structured and pre-determined strategy. The
first exploits the remarkable visual geometry of lenses in space- or
ground-based imaging data (e.g. MDS \citet{ratnatunga:99}, EGS
\citet{moustakas:07}, SL2S \citet{cabanac:07}, HAGGLeS
\citet{marshall:09}). The second is spectroscopic, and relies on the
observation of ``anomalous'' spectroscopic features.

This latter approach was pioneered by \citet{warren:96}, and has come
to spectacular fruition by the Sloan Lens ACS (SLACS) Survey
\citep{bolton:08}, which has confirmed some one hundred new strong
galaxy-galaxy lenses by \emph{Hubble} imaging follow-up to
spectroscopically identified candidates.  While the efficiency
afforded by the Sloan spectroscopic database is high, lenses have been
uncovered through regular ground-based spectroscopic observations
(e.g.~\citealt{bolton:06}).  In this paper we report on similarly
discovered lenses, in several distant galaxy clusters, which all
happen to be from the X-ray flux-limited Massive Cluster Survey (MACS)
sample of \citet{ebeling:01}. 

The observations are described in the following section, followed by a
more detailed presentation of the lenses and the available data.  We
end with a discussion of the lensing rates implied by these
discoveries, how other surveys may fare, and the general utility of
such lenses. All magnitudes are given in AB, and where relevant we
adopt a concordance cosmology. 

\begin{center}
\begin{table*}[t]
\begin{center}
%\scriptsize
\caption{\label{tab:data}}
\begin{tabular}{cccccccc}
\hline\hline
\multicolumn{1}{c}{RA} & 
\multicolumn{1}{c}{Dec} & 
\multicolumn{1}{c}{Cluster} & 
\multicolumn{1}{c}{$m_{\rm filter}$} &
\multicolumn{1}{c}{$\theta_{\rm E}$} & 
\multicolumn{1}{c}{$z_{\rm lens}$} & 
\multicolumn{1}{c}{$z_{\rm source}$}&
\multicolumn{1}{c}{Notes}\\
%%
\multicolumn{2}{c}{J2000} & 
\multicolumn{1}{c}{} & 
\multicolumn{1}{c}{AB} & 
\multicolumn{1}{c}{arcsec} & 
\multicolumn{1}{c}{} & 
\multicolumn{1}{c}{} & 
\multicolumn{1}{c}{}\\
\hline
00:25:23.487 & $-$12:19:42.8 & MACS\,J0025.4$-$1222  & $30.0$                  & $1.0\pm0.3$ & 0.582 & 2.047 & \\
17:20:06.416 & $+$35:35:26.3 & MACS\,J1720.3$+$3536 & $r_{\rm SDSS}=20.6$ & $1.0\pm0.3$ & 0.384 & 2.132 & \\
17:20:18.486 & $+$35:37:12.8 & MACS\,J1720.3$+$3536 & $r_{\rm SDSS}=19.1$ & $1.0\pm0.3$ & 0.385 & 2.787 & \\
21:29:19.973 & $-$07:42:18.2 & MACS\,J2129.4$-$0741 & $r_{\rm SDSS}=21.6$  & $1.0\pm0.3$ & 0.594 & 1.211 & AGN Source\\
\end{tabular}
\end{center}
\end{table*}
\end{center}


\section{Observations}

The data presented here are drawn from the ``cosmic chronometer''
project, which aims to make a new measurement of the expansion history
of the universe through the age-redshift relation.  To that end, many
galaxies have been spectroscopically targetted at high signal to
noise, for deriving precise stellar population ages
\citet{jimenez:04}.  Since the goal is to measure the upper age
envelope as a function of redshift, the targets are predominantly old,
red galaxies in galaxy clusters at $0.17<z<0.92$.  The full dataset is
presented in Stern et al.~(in prep).  All of the observations were
made with the Low Resolution Imaging Spectrometer (LRIS;
\citealt{oke:95}) on Keck~I.

\section{The Strong Lenses}

Brief preamble. Subsections on each lens, with data presented. With
the spec-only lenses, estimate the einstein radius based on the
emission features' positions. Dwell on the baby bullet lens, which has
HST data, and for which we can build a full lens model. Main features
etc.

\subsection{L0025-1219 in MACS\,J0025.4$-$1222} 

This galaxy cluster is an interesting system of two merging
subclusters \citep{bradac:08}, with some analogies to the original
Bullet Cluster (1E\,0657$-$56; \citealt{clowe:06},
\citealt{bradac:06}). 

This lens, properties, magnitude, spectroscopic
discovery figure, fidelity of observations and conclusions. Hubble
imaging details, give in a Table for conciseness. Magnitudes from
galfit modeling. Lens model parameters chosen, importance in terms of
degeneracies or robustness. Conclusions that can reasonably be drawn
from the lens modeling. How does this fit in with other work done in
this cluster?

\subsection{L1720+3535 in MACS\,J1720.3$+$3536}
 
Stuff on the spec detection of this one, einstein radius estimates,
particular features and whatnot. Is the extra convergence by the
cluster pretty clearly evident? Estimated position and distance from
center of cluster, based on where the target is. Some simple diagram
or plot of its position?

\subsection{L1720+3537 in MACS\,J1720.3$+$3536}

Similar to the previous section. Distance from central position, both
angular and projected physical. Estimated einstein radius. Does that
radius jive with the simple properties of the lens galaxy, or is the
extra convergence from the cluster pretty self-evident?

\subsection{L2129-0742 in MACS\,J2129.4$-$0741}

The archival HST imaging of this spectroscopically identified lens
candidate (program 9722) shows it unambiguously as a {\bf short-axis
  cusp} lens.  A simple lens model is calculated with gravlens
\citep{keeton:01} and is presented in [FIGURE/SHOWCASE]. By and by,
the time delays between the images are estimated to be YYY. From the
spectroscopy (Figure~\ref{fig:xx}), the lensed source is a Type~II
AGN, and [given brightness could be a lens-monitoring candidate of
particular interest, due to the brightness and to the many
spectroscopic features accessible in the optical window. 

It is interesting to note that observations (from program 10493), missed
covering this lens by less than an arc minute.  

\section{Discussion}

What good are such lenses? The fact that they are in galaxy clusters,
and particularly off-center, means that with high quality high
resolution imaging, they can provide otherwise inaccessible
information about the local surface density (the convergence) and the
orientation and strength of the local tidal field (the shear). Several
such bits of information could become an important third component in
the mass profile reconstruction of galaxy clusters, joining weak
lensing shear (through distortions of many background galaxies) and
strong lensing by the full cluster potential (cluster arcs).  

As a particular example, talk through the baby bullet cluster, and
need for synthesis of complementary tracers of the mass and mass
profile. 

By the nature of the parent survey, an accurate determination of the
lensing rate is not really possible with the current project, as would
be possible with a systematically undertaken ``complete'' survey such
as with SDSS \citep[e.g][]{dobler:08}.  Noting that the targets were
predominantly bright and luminous red early-type galaxies, the numbers
are XXXXX.  [Work through the probability numbers for
lenses. Considering the ``bias'' towards red luminous galaxies in the
targetted objects, what sort of surface density does this correspond
to?  Compare with the ``high-density'' numbers from the Treu SLACS
paper.]

This kind of work argues for the potential of large-scale
spectroscopic surveys towards the serendipitous discovery of strong
lenses.  Prospects for future lenses discovered spectroscopically in
similar high density environments, eg by BOSS (SDSS3), and WFMOS or
BigBOSS. 

Another lesson that may be taken from this work would be to encourage
much wider high-resolution observations around galaxy clusters.
[Especially if the work done shows that there is a pretty significant
convergence boost!].  [arguments towards supporting extensive
wide-field surveys of clusters, e.g. in the WFC3 era! good leverage
for multi cycle proposal.]. 

\acknowledgements

This work was carried out at Jet Propulsion Laboratory, California
Institute of Technology, under a contract with NASA.  LAM acknowledges
support by the NASA ATFP program.

\bibliographystyle{apj}
\bibliography{/Users/leonidas/Documents/fullbibdesk}

\end{document}

\message{ !name(clashresolved.tex) !offset(-340) }
